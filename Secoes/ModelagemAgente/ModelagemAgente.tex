\section{MODELAGEM EM OPERADORES AOS AGENTES}
\label{sec:modelagemAgentes}

Nesta seção é apresentada a modelagem realizada à representação computacional dos agentes e são descritos os atributos e valores considerados ao modelo.

Um agente $\chi(t)$ é definido espaço-temporalmente especificando-se como ocorre a transição do seu estado num intervalo de tempo $t$ e seu movimento no espaço, que é o ambiente computacional em que o agente é especificado. O estado do agente é especificado por meio do conjunto de atributos, como apresentado em (\ref{eq:especificacaoAgente}).
\begin{equation}
\label{eq:especificacaoAgente}
 \chi(t) \equiv \big(Q, S, I, L, X, Y, C, E \big)
\end{equation}
cujos significados dos identificadores dos atributos do estado interno do agente $\chi(t)$ são como:

\begin{itemize}
 \item \textbf{Quadra, $Q$:} Identificador da quadra que o agente $\chi(t)$ se encontra atualmente.
 \item \textbf{Sexo, $S$:} Identificador do sexo do agente $\chi(t)$, podendo ser macho ou fêmea.
 \item \textbf{Faixa etária, $I$:} Identificador da faixa etária do agente $\chi(t)$, podendo ser criança, jovem, adulto ou idoso.
 \item \textbf{Lote, $L$:} Identificador do lote que o agente $\chi(t)$ se encontra atualmente.
 \item \textbf{Latitude, $X$:} Identificador da latitude da posição que o agente $\chi(t)$ se encontra atualmente.
 \item \textbf{Longitude, $Y$:} Identificador da longitude da posição que o agente $\chi(t)$ se encontra atualmente.
 \item \textbf{Contador de transições de estados, $C$:} Contador utilizado na transição de estados do agente $\chi(t)$.
 \item \textbf{Estado, $E$:} Identificador da saúde do agente $\chi(t)$, podendo ser suscetível, exposto, infectante e recuperado.
\end{itemize}

A Seção \ref{sec:modelagemBitstring} apresenta a modelagem \textit{bitstring} empregada à representação e manipulação dos atributos dos agentes e discute detalhes da implementação computacional realizada.

\newpage