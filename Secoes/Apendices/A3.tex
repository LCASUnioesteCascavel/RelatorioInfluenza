\section{APÊNDICE III - CÓDIGO-FONTE DOS SCRIPTS AUXILIARES PYTHON}

\subsection{Scripts à Geração de Gráficos} 

\subsubsection{Arquivo graficos.py}

Este \textit{script} é utilizado à geração de gráficos de quantidades ao longo do tempo de simulação para a população de agentes humanos. Os gráficos gerados são salvos em arquivos \textit{.png}, fornecendo as opções de saídas da população total ou por quadras. Este \textit{script} utiliza as classes definidas no arquivo \textit{graficos\_h.py}. As classes utilizam a biblioteca \textit{Python Matplotlib} para a confecção dos gráficos.

\lstinputlisting[title=graficos.py, captionpos=b, language=Python]{Codigos/Apendices/A3/graficos.py}

\newpage

\subsubsection{Arquivo graficos\_h.py}

Este \textit{script} define a classe responsável pela geração dos gráficos das populações dos agentes humanos ao longo do tempo de simulação. São gerados os gráficos por saúdes, faixas etárias, sexos, sorotipos, saúdes e faixas etárias, saúdes e sorotipos, saúdes e sexos, faixas etárias e sorotipos, faixas etárias e sexos e sorotipos e sexos.

\lstinputlisting[title=grafico\_h.py, captionpos=b, language=Python]{Codigos/Apendices/A3/grafico_h.py}

\newpage

\subsection{Outros Scripts} 

\subsubsection{Arquivo bitstring.py}

Este \textit{script} é utilizado no cálculo das quantidades de \textit{bits} anteriores, máscaras positivas e negativas à cada campo, que são empregadas à modelagem \textit{bitstring} dos agentes humanos, que é apresentada na Seção \ref{sec:modelagemBitstring}. São fornecidos como entrada ao \textit{script} os tamanhos dos campos e as quantidades de palavras e \textit{bits} por palavra. O \textit{script} exibe no terminal as quantidades de \textit{bits} anteriores e as máscaras positivas e negativas em decimal correspondentes às entradas fornecidas. 

\lstinputlisting[title=bitstring.py, captionpos=b, language=Python]{Codigos/Apendices/A3/bitstring.py}

\newpage
