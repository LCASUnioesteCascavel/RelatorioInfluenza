\sloppy
\begin{frontmatter}
\title{UM MODELO MULTIAGENTE \textit{BITSTRING} EM GPU À SIMULAÇÃO DA PROPAGAÇÃO DA INFLUENZA NA CIDADE DE CASCAVEL/PR}
\author[unioeste]{Rogério Luis Rizzi\corref{cor1}}
\ead{rogeriorizzi@hotmail.com}
\author[fgv]{Flávio Coelho}
\author[unioeste]{Claudia Brandelero Rizzi}
\author[unioeste]{Guilherme Galante}
\author[unioeste]{Wesley Luciano Kaizer}
\cortext[cor1]{Corresponding author. Tel.:+55 45 99730557.}
\address[unioeste]{Centro de Ciências Exatas e Tecnológicas, Universidade Estadual do Oeste do Paraná - Cascavel/PR, Brasil.}
\address[fgv]{EMAp - Escola de Matemática Aplicada, Fundação Getúlio Vargas - Rio de Janeiro/RJ, Brasil.}

\begin{abstract}
Neste relatório técnico é apresentado um modelo computacional para simular a propagação da Influenza em ambientes computacionais georreferenciados. Na abordagem empregada utiliza-se sistemas multiagente baseados em modelos, sendo um agente definido como uma entidade capaz de perceber e interagir no ambiente onde está situado. Especifica-se como ocorre sua movimentação no espaço, o contato com outros agentes e a transição do seu estado num intervalo de tempo, considerando seu atual estado e posição no ambiente. São apresentadas as especificações das modelagens \textit{bitstring} empregadas com o objetivo de reduzir o consumo de memória para o armazenamento dos agentes. São apresentados aspectos específicos relativos à implementação e a paralelização do modelo em linguagens de programação. Conclui-se elencando os principais resultados obtidos por meio da execução de simulações, apresentando saídas gráficas das populações ao longo do tempo, consumos de memória, tempos de execução e saídas espaciais georreferenciadas nos distintos ambientes computacionais utilizados. 
\end{abstract}

\begin{keyword}
Influenza. Modelagem compartimental. Sistema multiagente baseado em modelo. Modelagem \textit{bitstring}. CUDA. GIS.
\end{keyword}

\end{frontmatter}

\newpage
